%%%%%%%%%%%%%%%%%%%%%%%%%%%%%%%%%%%%%%%%%%%%%%%%%%%%%%%%%%%%%%%%%%%%%%
% writeLaTeX Example: A quick guide to LaTeX
%
% Source: Dave Richeson (divisbyzero.com), Dickinson College
% 
% A one-size-fits-all LaTeX cheat sheet. Kept to two pages, so it 
% can be printed (double-sided) on one piece of paper
% 
% Feel free to distribute this example, but please keep the referral
% to divisbyzero.com
% 
%%%%%%%%%%%%%%%%%%%%%%%%%%%%%%%%%%%%%%%%%%%%%%%%%%%%%%%%%%%%%%%%%%%%%%
% How to use writeLaTeX: 
%
% You edit the source code here on the left, and the preview on the
% right shows you the result within a few seconds.
%
% Bookmark this page and share the URL with your co-authors. They can
% edit at the same time!
%
% You can upload figures, bibliographies, custom classes and
% styles using the files menu.
%
% If you're new to LaTeX, the wikibook is a great place to start:
% http://en.wikibooks.org/wiki/LaTeX
%
%%%%%%%%%%%%%%%%%%%%%%%%%%%%%%%%%%%%%%%%%%%%%%%%%%%%%%%%%%%%%%%%%%%%%%

\documentclass[10pt,landscape]{article}
\usepackage{amssymb,amsmath,amsthm,amsfonts}
\usepackage{multicol,multirow}
\usepackage{calc}
\usepackage{ifthen}
\usepackage[landscape]{geometry}
\usepackage[colorlinks=true,citecolor=blue,linkcolor=blue]{hyperref}


\ifthenelse{\lengthtest { \paperwidth = 11in}}
    { \geometry{top=.5in,left=.5in,right=.5in,bottom=.5in} }
	{\ifthenelse{ \lengthtest{ \paperwidth = 297mm}}
		{\geometry{top=1cm,left=1cm,right=1cm,bottom=1cm} }
		{\geometry{top=1cm,left=1cm,right=1cm,bottom=1cm} }
	}
\pagestyle{empty}
\makeatletter
\renewcommand{\section}{\@startsection{section}{1}{0mm}%
                                {-1ex plus -.5ex minus -.2ex}%
                                {0.5ex plus .2ex}%x
                                {\normalfont\large\bfseries}}
\renewcommand{\subsection}{\@startsection{subsection}{2}{0mm}%
                                {-1explus -.5ex minus -.2ex}%
                                {0.5ex plus .2ex}%
                                {\normalfont\normalsize\bfseries}}
\renewcommand{\subsubsection}{\@startsection{subsubsection}{3}{0mm}%
                                {-1ex plus -.5ex minus -.2ex}%
                                {1ex plus .2ex}%
                                {\normalfont\small\bfseries}}
\makeatother
\setcounter{secnumdepth}{0}
\setlength{\parindent}{0pt}
\setlength{\parskip}{0pt plus 0.5ex}
% -----------------------------------------------------------------------

\title{Attitude Description Sets: Cheatsheet}

\begin{document}
	
\begin{center}
	\Large{\textbf{Attitude Description Sets: A Cheatsheet}} \\
\end{center}

\begin{multicols}{3}
	\setlength{\premulticols}{1pt}
	\setlength{\postmulticols}{1pt}
	\setlength{\multicolsep}{1pt}
	\setlength{\columnsep}{2pt}
	\raggedright
	\footnotesize

\section{The Attitude Description Problem}


\section{Parameter Sets}

\subsection{Directional Cosine Matrix}
% Definition

\subsubsection{Addition/Subtraction}

\subsubsection{Differential Kinematic Relation}

\subsection{Principle Rotation Vector}

\subsection{Euler Angles}

\subsection{Quaternions/Euler Parameters}

\subsubsection{Addition/Subtraction}


Addition:
%\begin{center}
$ \left[ FN(\beta) \right] = \left[ FB(\beta'') \right] \left[ BN(\beta') \right] $
%\end{center}

$
	\left( \begin{array}{c}
		\beta_0 \\ \beta_1 \\ \beta_2 \\ \beta_3
	\end{array} \right)
	=
	\left[ \begin{array}{cccc}
		\beta_0'' & -\beta_1'' & -\beta_2'' & -\beta_3'' \\
		\beta_1'' &  \beta_0'' &  \beta_3'' & -\beta_2'' \\
		\beta_2'' & -\beta_3'' &  \beta_0'' &  \beta_1'' \\
		\beta_3'' &  \beta_2'' & -\beta_1'' &  \beta_0'' 
	\end{array} \right]
	\left( \begin{array}{c}
		\beta_0' \\ \beta_1' \\ \beta_2' \\ \beta_3'
	\end{array} \right) \\
	=
	\left[ \begin{array}{cccc}
		\beta_0' & -\beta_1' & -\beta_2' & -\beta_3' \\
		\beta_1' &  \beta_0' &  \beta_3' & -\beta_2' \\
		\beta_2' & -\beta_3' &  \beta_0' &  \beta_1' \\
		\beta_3' &  \beta_2' & -\beta_1' &  \beta_0' 
	\end{array} \right]
	\left( \begin{array}{c}
		\beta_0'' \\ \beta_1'' \\ \beta_2'' \\ \beta_3''
	\end{array} \right)
$	


Subtraction:
%\begin{center}
$ \left[ BN(\beta') \right] = \left[ FB(\beta'')^T \right] \left[ FN(\beta) \right] $
%\end{center}

$
	\left( \begin{array}{c}
		\beta_0' \\ \beta_1' \\ \beta_2' \\ \beta_3'
	\end{array} \right)
	=
	\left[ \begin{array}{cccc}
		 \beta_0'' &  \beta_1'' &  \beta_2'' &  \beta_3'' \\
		-\beta_1'' &  \beta_0'' & -\beta_3'' &  \beta_2'' \\
		-\beta_2'' &  \beta_3'' &  \beta_0'' & -\beta_1'' \\
		-\beta_3'' & -\beta_2'' &  \beta_1'' &  \beta_0'' 
	\end{array} \right]
	\left( \begin{array}{c}
		\beta_0 \\ \beta_1 \\ \beta_2 \\ \beta_3
	\end{array} \right) \\
$	

\subsubsection{Differential Kinematic Relation}

$
	\left( \begin{array}{c}
		\dot{\beta_0} \\ \dot{\beta_1} \\ \dot{\beta_2} \\ \dot{\beta_3}
	\end{array} \right)
	=
	\frac{1}{2}
	\left[ \begin{array}{cccc}
		\beta_0 & -\beta_1 & -\beta_2 & -\beta_3 \\
		\beta_1 &  \beta_0 & -\beta_3 &  \beta_2 \\
		\beta_2 &  \beta_3 &  \beta_0 & -\beta_1 \\
		\beta_3 & -\beta_2 &  \beta_1 &  \beta_0 
	\end{array} \right]
	\left( \begin{array}{c}
		0 \\ \omega_1 \\ \omega_2 \\ \omega_3
	\end{array} \right)
$

\subsection{Gibbs Vector/Classical Rodriguez Parameters}

\subsubsection{Definition}

Euler Parameter relations: \\
$ q_i = \frac{\beta_i}{\beta_0} $

inversely,

$ \beta_0 = \frac{1}{1 + \textbf{q}^T \textbf{q}}; 
  \beta_i = \frac{q_i}{1 + \textbf{q}^T \textbf{q}} $

Principle Rotation Vector relations: \\

$ \textbf{\sigma} = tan \left( \frac{\Phi}{2} \right) \hat{\textbf{e}}; \\
  \textbf{\sigma} \approx \left( \frac{\Phi}{2} \right) \hat{\textbf{e}} $
  
\begin{align*}
	\left[ C \right] &= \frac{1}{1 + \textbf{q}^T \textbf{q}}
	\left[ \begin{array}{ccc}
		1 + q_1^2 - q_2^2 - q_3^2 & 2(q_1 q_2 + q_3) & 2(q_1 q_3 - q_2) \\
		2(q_1 q_2 - q_3) & 1 - q_1^2 + q_2^2 - q_3^2 & 2(q_2 q_3 + q_1) \\
		2(q_1 q_3 + q_2) & 2(q_2 q_3 - q_1) & 1 - q_1^2 - q_2^2 + q_3^2
	\end{array} \right] \\
	%
	&= \frac{1}{1 + \textbf{q}^T \textbf{q}}
	\left( \left( 1 - \textbf{q}^T \textbf{q} \right) \left[ I_{3x3} \right] + 2\textbf{q}\textbf{q}^T - 2 \left[ \tilde{\textbf{q}}\right] \right)
\end{align*}


\subsection{Modified Rodriguez Parameters}

\subsubsection{Definition}

Euler Parameter relations: \\
$ \sigma_i = \frac{\beta_i}{1 + \beta_0} $

inversely,

$ \beta_0 = \frac{1 - \sigma^2}{1 + \textbf{\sigma}^T \textbf{\sigma}}; 
  \beta_i = \frac{2 \sigma_i}{1 + \textbf{\sigma}^T \textbf{\sigma}} $

Principle Rotation Vector relations: \\

$ \textbf{\sigma} = tan \left( \frac{\Phi}{4} \right) \hat{\textbf{e}}; \\
  \textbf{\sigma} \approx \left( \frac{\Phi}{4} \right) \hat{\textbf{e}}
$

Classical Rodriguez Parameter relations: \\

$ \textbf{q} = \frac{2 \textbf{\sigma}}{1 - \sigma^2};
  \textbf{\sigma} = \frac{\textbf{q}}{1 + \sqrt{1 + q^2}} $

\subsubsection{Addition/Subtraction}

Addition: \\
%\begin{center}
$ \left[ FN(\beta) \right] = \left[ FB(\beta'') \right] \left[ BN(\beta') \right] $ \\
%\end{center}

$
\textbf{\sigma} = \frac{(1 - |\sigma'|^2)\sigma'' + (1 - |\sigma''|^2)\sigma' - 2\sigma'' \times \sigma'}{1 + |\sigma'|^2|\sigma''|^2 - 2\sigma'\cdot \sigma''}
$

Subtraction: \\
%\begin{center}
$ \left[ BN(\beta') \right] = \left[ FB(\beta'')^T \right] \left[ FN(\beta) \right] $ \\
%\end{center}

$
\textbf{\sigma}'' = \frac{(1 - |\sigma'|^2)\sigma - (1 - |\sigma|^2)\sigma' + 2\sigma \times \sigma'}{1 + |\sigma'|^2|\sigma|^2 + 2\sigma'\cdot \sigma}
$

\subsubsection{Differential Kinematic Relation}

$
\dot{\textbf{\sigma}} =
\frac{1}{4}
\left[ \begin{array}{ccc}
	1 - \sigma^2 + 2\sigma_1^2 & 2(\sigma_1 \sigma_2 - \sigma_3) & 2(\sigma_1 \sigma_3 + \sigma_2) \\
	2(\sigma_1 \sigma_2 + \sigma_3) &  1 - \sigma^2 + 2\sigma_2^2 & 2(\sigma_2 \sigma_3 - \sigma_1) \\
	2(\sigma_1 \sigma_3 - \sigma_2) &  2(\sigma_2 \sigma_3 + \sigma_1) &  1 - \sigma^2 + 2\sigma_3^2 
\end{array} \right]
\left( \begin{array}{c}
	\omega_1 \\ \omega_2 \\ \omega_3
\end{array} \right)
$
%$
%\dot{\textbf{\sigma}} =
%\frac{1}{4} \left[ \left( 1 - \sigma^2 \right) \left[ I_{3x3} \right] + 2 \left[ \tilde{\sigma} \right] + 2\textbf{\sigma \sigma^T} \right] \textbf\{omega}
%$

% \subsection{Stereographic Projection}

\section{Resources}

\end{multicols}
\end{document}
